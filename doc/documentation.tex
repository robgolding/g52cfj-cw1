\documentclass[a4paper,11pt]{article}

\usepackage{url}
\usepackage{graphicx}
\usepackage{enumerate}
\usepackage{amsmath}
\usepackage{float}
\usepackage{longtable}
%\usepackage{fullpage}
\usepackage{pstricks}
\usepackage{tikz}
\usepackage[absolute]{textpos}
\usepackage{import}
\usepackage{subfigure}
\usepackage{setspace}

\title{G52CFJ C++ for Java Programmers \\ Coursework Documentation}
\author{Robert J. Golding (rjg08u)} \date{\today}

% Dutch style paragraph formatting
\setlength{\parskip}{1.3ex plus 0.2ex minus 0.2ex}
\setlength{\parindent}{0pt}

%\doublespacing
\onehalfspacing

\begin{document}
    \maketitle

    \section{Overview}

    The game is intended to be a clone of the popular Pacman game. Though it
    does not implement all of the functionality of the official game, all of
    the coursework requirements are fulfilled.

    \section{Usage}

    The game starts with an initialisation screen, requiring the user to press
    the space bar to start. Once the game is running, pressing space again
    pauses. Pressing the escape key at any point exits the game.

    The aim of the game is to collect all of the ``pellets'' without being
    caught by the enemies. The larger pellets are ``power-ups'' which work in
    the same way as the original Pacman game---allowing the player to catch
    enemies, sending them back to their original starting location.

    Once all the pellets have been collected, the level is complete. Each
    pellet scores the player 10 points. A power-up pellet is worth 50 points.
    The power-up mode lasts for 5 seconds, and whilst in this mode normal
    pellets score double the normal amount (20 points).

    To move the player, use the arrow keys on the keyboard. Once moving, the
    player will continue to move in the same direction until a wall is reached,
    when it will stop. Pressing a key to move in a direction which is
    impossible due to a wall will delay that action until it is next available,
    in much the same way as the original Pacman game.

    \section{Requirements}

    \subsection{Draw an Appropriate Background}
    \subsection{Have Moving Objects}
    \subsection{Have Interaction Between the Objects \& Background}
    \subsection{Provide User Interaction}
    \subsection{Provide AI-Controlled Objects}
    \subsection{Load Data From Files}
    \subsection{Save and Load Information}
    \subsection{Display Status Information on the Screen}
    \subsection{Support Different States}

    \section{Additional Comments}

\end{document}
